\documentclass[calcdimensions,landscape,letterpaper]{powersem}
\usepackage{color}
\usepackage[pdftex]{thumbpdf} % For thumbnails
\usepackage[pdftex]{hyperref} % For links.
\usepackage[display,coloremph,whitebackground]{texpower} % colormath
\usepackage{fixseminar}
\usepackage{tpslifonts}
\usepackage[pdftex,final]{graphicx} % For including graphics.
\usepackage[utf8]{inputenc}
\usepackage{minted}
\usepackage{amsmath}
\usepackage{amssymb}
\usepackage{eurosym}
\usepackage{booktabs} % rules in tables
\usepackage{multirow} % multi-row cells
\usepackage{rotating} % text rotation

\title{A Short Introduction to OpenGL}
\author{Jan Wedekind}
\date{Thursday, May 30th 2024}

\DeclareGraphicsExtensions{.jpg,.pdf,.png}
\pdfcompresslevel=9

\hypersetup{
   pdftitle          = {\thetitle},
   pdfsubject        = {After Catchup Presentation 30th May 2024},
   pdfauthor         = {Jan Wedekind},
   pdfkeywords       = {opengl, graphics, software},
   pdfcreator        = {okular},
   pdfproducer       = {LaTeX with hyperref and thumbpdf},
   bookmarksopen     = false,
   bookmarksnumbered = true,
   colorlinks        = true,
   pdfstartpage      = {1},
   pdfpagemode       = {FullScreen}
}

\DeclarePanel{top}{
  \begin{picture}(0,0)
    \put(-3, -293){\resizebox*{\pdfpagewidth}{\pdfpageheight}{\includegraphics{slide.png}}}
    \put(-10,-20){\parbox[c]{.9\textwidth}{\center\large\bf\thecurrentheading}}
  \end{picture}
}

\DeclarePanel{bottom}{
  \begin{picture}(0,0)
    \put(0,15){\parbox[c]{.98\pdfpagewidth}{\tiny\thedate\hfill\theslide/22}}
  \end{picture}
}

\slidesmag{4}
\backgroundstyle{none}
\slideframe{none}
\pagestyle{empty}

\mklength{\slideleftmargin}{-2cm}
\mklength{\sliderightmargin}{-2cm}
\mklength{\slidetopmargin}{2.0cm}
\mklength{\slidebottommargin}{1.5cm}

\renewcommand{\currentpagevalue}{\value{slide}}
\newcommand{\thecurrentheading}{}
\newcommand{\heading}[1]{\renewcommand{\thecurrentheading}{#1}}
\newcommand{\subheading}[1]{\concept{#1}}

\begin{document}

\begin{slide}
  \pdfbookmark[1]{\thetitle}{title}
  \heading{\ }
  \begin{center}
    \maketitle
  \end{center}
\end{slide}

\begin{slide}
  \pdfbookmark[1]{Motivation}{motivation}
  \heading{Motivation}
  \begin{center}
    \begin{minipage}[c]{.5\textwidth}
      \begin{center}
        OpenGL
      \end{center}
    \end{minipage}
  \end{center}
\end{slide}

% Vulkan is more modern but OpenGL is easier to learn and knowledge is transferable.
% OpenGL runs on Windows, Linux, MacOS (stuck at 4.1)
% OpenGL versions
% OpenGL superbible
% Minimal OpenGL pipeline: Vertex Shader, Fragment Shader (page 47)
% GLFW to handle window and events
% clear color
% compile and link program rendering triangle with constant color
% Texture mapping
% vertex array objects
% render triangle with different colors
% normalised device coordinates
% Depth buffer, Reversed-z rendering
% uniform variable
% Projection matrix
% Transformation matrix
% Culling
% Blending
% Phong shading
% Normal mapping
% Tessellation
% Mipmapping
% Shadow mapping
% HDR bloom
% edge detection
% fog

\begin{slide}
  \pdfbookmark[1]{glsl}{try-glsl}
  \heading{GLSL}
  \begin{center}
    \begin{minted}{glsl}
#version 410 core
int main()
{
  fragColor = vec3(1, 1, 1);
}
    \end{minted}
  \end{center}
\end{slide}

\end{document}
